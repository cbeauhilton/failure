\section{Results}%\% Number of characters in abstract: 5671. 
\% ASH character limit is 3800. 
\% You are over by 1871 non-whitespace characters.


Background

Myelodysplastic syndromes (MDS) and chronic myelomonocytic leukemia (CMML) are mainly diagnosed based on morphological changes in the bone marrow. The diagnosis can be challenging in patients (pts) with pancytopenia with minimal dysplasia, and is subject to inter-observer variability. Somatic mutations can be identified in either disease but no genes, in isolation or in combination, are specific for disease phenotype.

We developed a geno-clinical model that uses mutational data, peripheral blood values, and clinical variables to predict an MDS vs. CMML diagnosis in pts who presented with cytopenias, in the absence of bone marrow biopsy results.


Methods

We combined genomic and clinical data from 1897 pts treated at our institution (593) and the Munich Leukemia Laboratory (1304). Pts were diagnosed with MDS or CMML according to 2008 WHO criteria. Diagnosis of MDS and CMML was confirmed by independent hematopathologists that were not associated with the study. A genomic panel of 40 genes commonly mutated in myeloid malignancies was included. The initial cohort was randomly (computer generated) divided into learner (80\%) and validation (20\%) cohorts. Multiple machine learning algorithms were applied to predict the phenotype. Feature extraction algorithms were used to extract genomic/clinical variables that impacted the algorithm decision and to visualize the impact of each variable on phenotype. Prediction performance was evaluated according to the area under the curve of the receiver operator characteristic (ROC-AUC) and confusion/accuracy matrices.


Results

Of 2471 pts included, 1306 pts had MDS, 478 pts had CMML, 223 pts had ICUS, 107 pts had CCUS, 99 pts had PMF, 90 pts had ET, 89 pts had MDS_MPN, and 79 pts had PV.The median age for the entire cohort was 71 years (range, 9-102). 38\% were female. The median white blood cell count (WBC) was 3.2x10^9/L (range, 0.00-179), absolute monocyte count (AMC) 0.21x10^9/L (range, 0-96), absolute lymphocyte count (ALC) 0.88x10^9/L (range, 0-357), absolute neutrophil count (ANC) 0.60x10^9/L (range, 0-170), and hemoglobin (Hgb)  10.50 (range, 3.9-24.0). 

The most commonly mutated genes in all pts were: TET2 (28.0\%), ASXL1 (22.9\%), SF3B1 (15.2\%), SRSF2 (13.6\%), JAK2 (11.1\%), DNMT3A (9.0\%), RUNX1 (9.0\%), U2AF1 (6.1\%), CBL (5.7\%), STAG2 (5.1\%), EZH2 (4.7\%), ZRSR2 (4.5\%). In CCUS, they were: TET2 (45.8\%), SRSF2 (24.3\%), ASXL1 (23.4\%), DNMT3A (23.4\%), U2AF1 (8.4\%), RUNX1 (5.6\%), SF3B1 (4.7\%), ZRSR2 (3.7\%), CBL (2.8\%), IDH2 (2.8\%), TP53 (2.8\%), IDH1 (1.9\%). In CMML, they were: TET2 (51.0\%), ASXL1 (42.9\%), SRSF2 (24.7\%), RUNX1 (17.2\%), CBL (16.5\%), KRAS (11.5\%), NRAS (11.3\%), EZH2 (8.6\%), JAK2 (6.3\%), U2AF1 (4.8\%), SF3B1 (4.4\%), NF1 (3.6\%). In ET, they were: JAK2 (44.4\%), TET2 (13.3\%), DNMT3A (7.8\%), ASXL1 (6.7\%), SRSF2 (3.3\%), CUX1 (2.2\%), SF3B1 (2.2\%), DHX29 (1.1\%), GATA2 (1.1\%), IDH1 (1.1\%), PRPF8 (1.1\%), TP53 (1.1\%). In ICUS, they were: APC (0.0\%), ASXL1 (0.0\%), BCOR (0.0\%), BCORL1 (0.0\%), C7ORF55 (0.0\%), CBL (0.0\%), CCDC42B (0.0\%), CDH23 (0.0\%), CEBPA (0.0\%), CFTR (0.0\%), CSF1R (0.0\%), CUX1 (0.0\%). In MDS, they were: TET2 (26.1\%), SF3B1 (23.6\%), ASXL1 (20.8\%), SRSF2 (12.9\%), DNMT3A (11.8\%), RUNX1 (9.8\%), STAG2 (8.6\%), U2AF1 (7.7\%), ZRSR2 (7.0\%), TP53 (6.7\%), BCOR (4.9\%), EZH2 (4.7\%). In MDS_MPN, they were: SF3B1 (39.3\%), JAK2 (37.1\%), TET2 (20.2\%), ASXL1 (12.4\%), EZH2 (9.0\%), SMC3 (6.7\%), BCOR (5.6\%), CBL (5.6\%), DNMT3A (5.6\%), U2AF1 (5.6\%), CEBPA (4.5\%), SETBP1 (4.5\%). In PMF, they were: JAK2 (66.7\%), ASXL1 (43.4\%), SRSF2 (17.2\%), TET2 (12.1\%), DNMT3A (11.1\%), U2AF1 (11.1\%), EZH2 (6.1\%), SF3B1 (5.1\%), TP53 (5.1\%), CUX1 (4.0\%), GATA2 (3.0\%), NRAS (3.0\%). In PV, they were: JAK2 (93.7\%), TET2 (21.5\%), DNMT3A (7.6\%), ASXL1 (5.1\%), SRSF2 (2.5\%), BCORL1 (1.3\%), DDX54 (1.3\%), IDH1 (1.3\%), SETBP1 (1.3\%), TP53 (1.3\%), ZRSR2 (1.3\%), APC (0.0\%).

The median total number of mutations/sample was 2.0 (range 0-27) for all pts, 1.0 (range 1-4) for CCUS, 2.0 (range 0-8) for CMML, 1.0 (range 0-4) for ET, 0.0 (range 0-0) for ICUS, 2.0 (range 0-27) for MDS, 2.0 (range 0-10) for MDS_MPN, 2.0 (range 0-6) for PMF and 1.0 (range 0-3) for PV.


A set of 108 genomic/clinical variables were evaluated and several feature extraction algorithms were used to identify the variables that have the most significant impact on the algorithm's decision. These variables included: (CCUS) ANC, ABC, Number of mutations, Plt, AMC, AEC, JAK2, Hgb, Age, DNMT3A, WBC, TET2, U2AF1, ALC, ASXL1, EED, DNMT3A VUS, DDX54 VUS, DHX29 VUS, DHX29; (CMML) AMC, AEC, WBC, ANC, ABC, ALC, Number of mutations, Plt, Hgb, Age, ASXL1, TET2, JAK2, SF3B1, RUNX1, SRSF2, BCOR, ETV6 VUS, DHX29, DHX29 VUS; (ET) ALC, Plt, Hgb, WBC, Age, AMC, AEC, ANC, ABC, Number of mutations, JAK2, CUX1, TET2, ETV6, DDX54 VUS, DHX29, DHX29 VUS, DNMT3A, DNMT3A VUS, EED; (ICUS) Number of mutations, ANC, ABC, AEC, Hgb, WBC, ALC, AMC, Plt, Age, Female gender, JAK2, TET2, DNMT3A, ERBB4, EED, DNMT3A VUS, DDX54 VUS, DHX29 VUS, DHX29; (MDS) ANC, Plt, AMC, Hgb, WBC, ALC, ABC, JAK2, Age, AEC, Number of mutations, SRSF2, SF3B1, STAG2, Female gender, TET2, GATA2, ERBB4, DDX54, DDX54 VUS; (MDS_MPN) Hgb, ALC, Plt, AMC, ANC, JAK2, Age, Number of mutations, AEC, WBC, ASXL1, ABC, SF3B1, EZH2, Female gender, DNMT3A, SRSF2, U2AF1, BCOR, NF1; (PMF) JAK2, ABC, AMC, AEC, Hgb, ALC, ANC, Age, Plt, WBC, ASXL1, U2AF1, TET2, Female gender, Number of mutations, PHF6, SRSF2, DDX41 VUS, DDX54, DDX54 VUS; (PV) Hgb, ALC, JAK2, WBC, Plt, AMC, Age, AEC, ABC, BCORL1, ANC, Number of mutations, APC VUS, DDX54 VUS, DHX29 VUS, DNMT3A, DNMT3A VUS, EED, ERBB4, ETV6 (Figure 1).

When applying the model to the validation cohort, the ROC-AUC was .98 with an accuracy of 94\%, with other statistical values as follows: specificities CMML 93\%, MDS 96\%; sensitivities CMML 96\%, MDS 93\%; positive predictive values CMML 84\%, MDS 98\%; negative predictive values CMML 98\%, MDS 84\%.

Individual pt data can also be entered into the model, with a probability of whether the diagnosis is MDS vs. CMML provided along with the impact of each variable on the decision, as shown in Figure 1.

When the analysis was restricted to mutations only, the accuracy of the model dropped dramatically (77\%, ROC-AUC .85).


Conclusions

We propose a novel approach using interpretable, individualized modeling to predict MDS vs. CMML phenotypes based on genomic and clinical data without the need for bone marrow biopsy data. This approach can aid clinicians and hematopathologists when encountering pts with cytopenias and a diagnosis suspicious for MDS vs. CMML. The model also provides feature attributions that allow for quantitative understanding of the complex interplay among genotype, clinical variables, and phenotype.


